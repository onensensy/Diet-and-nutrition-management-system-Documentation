\documentclass{article}
\usepackage{hyperref}
\usepackage[table,xcdraw]{xcolor}
\usepackage{booktabs} %Allow tables to be created well
\usepackage{graphicx}
\usepackage{float} % To make table appear where you place its code
\usepackage[margin=1in]{geometry} %border Margins

\bibliographystyle{unsrt}

\begin{document}

\title{\textbf{TOPIC:  DIET AND NUTRITION MANAGEMENT SYSTEM} }
\author{}
% \date{October, 2022}
\maketitle 
\begin{center}


\begin{center}
\textbf{COLLEGE OF COMPUTING AND INFORMATION SCIENCES\\ SCHOOL OF COMPUTING AND INFORMATICS TECHNOLOGY}\\
\end{center}


\begin{table}[H]
\centering
\resizebox{\columnwidth}{!}{%
\begin{tabular}{|l|l|l|l|}
\hline
\textbf{NAME} & \textbf{REG NO} & \textbf{STUDENT EMAIL ADDRESS} & \textbf{CONTACT NO} \\ \hline
MUHUMUZA VICTOR IAN & 20/U/524/PS & {\color[HTML]{000000} {victor.muhumuza@students.mak.ac.ug}} & 0761-656330 \\ \hline
WAKOKO SIMON PETER & 20/U/3512/PS & {\color[HTML]{000000} {simon.wakoko@students.mak.ac.ug}} & 0706-565245 \\ \hline
KIKOMEKO PETER GRACE & 20/U/3578/PS & {\color[HTML]{000000} { gracekikomeko@gmail.com}} & 0756-837136 \\ \hline
NAMBOOZE RACHEAL & 20/U/7994/EVE & {\color[HTML]{000000} {racheal.nambooze@students.mak.ac.ug}} & 0755-868603 \\ \hline
ONEN SAM SENSY & 20/U/3502/PS & {\color[HTML]{000000} {sensyonen@gmail.com}} & 0782-150448 \\ \hline
\end{tabular}%
}
\end{table}


\begin{center}
\textbf{Department of Information Systems}\\
\textbf{School of Computing and Informatics Technology}
\end{center}

\begin{center}
\textbf{A Proposal submitted to the College of Computing and Information Sciences in Partial fulfilment of the Requirements for the Award of a Degree of Bachelors of Information Systems and Technology of Makerere University. Option: Systems Development}
\end{center}


\textbf{Supervisor:} Mr. Bitwire Albert George \\
\textbf{Email:} bitwire.albert@gmail.com \hspace{1cm} \textbf{Contact:} 0773-095119 \\
\textbf{Submission date:} \date{\today}

\textbf{Option:} Systems Development
\end{center}

\newpage

\section{Introduction}
\label{Introducion}
Information systems are concerned with data capture, storage, analysis, and retrieval. These are vital to assist decision-making in a short time frame, potentially allowing decisions to be made and practices to be action-ed in real-time.  In the context of diet and nutrition management, nutritional health systems allow for accurate and efficient analysis of food ingredients as well as patient nutrient intake which assists in ensuring that selected food combinations offer the desired nutrients for a particular meal or diet order.


\subsection{Background}
Dietary deficiency is more accountable for the overall death toll on the global scale than other factors such as tobacco, high blood pressure, or any other health risk, as indicated in the new scientific study. “Poor diet is an equal opportunity killer,” \cite{healthdata}. Low levels of healthy food consumption, such as whole grains, in contrast to too many unhealthy foods, including sweetened beverages, account for one in every five deaths globally, “We are what we eat and risks affect people across a range of demographics, including age, gender, and economic status,” \cite{healthdata}.

Poor diets were behind the 10.9 million deaths, or 22\% of all deaths among adults in 2017, with cardiovascular disease (CVD) as the foremost cause, trailed by cancers and diabetes. They also resulted in 255 million disability-adjusted life years (DALYs), equalling the sum of years lost and years lived with disability. Poor diet statistically represents 16\% of all DALYs among adults globally. The findings of the study dictate that while the impact of individual dietary factors varies across countries, three dietary factors – low intake of whole grains, as well as fruits, and high consumption of sodium – accounted for more than 50\% of diet-related deaths and 66\% of DALYs. The other 50\% of death and 34\% of DALYs were tied to high consumption of red meat, processed meats, sugar-sweetened beverages, and trans fatty acids among other foods. There is an urgent and compelling need for changes in the various sectors of the food production cycle, such as growing, processing, packaging, and marketing \cite{healthdata}.

Uganda has shown limited progress toward achieving the diet-related non-communicable disease (NCD) targets. Furthermore, it has shown no progress towards achieving the target for obesity, with an estimated 10.4\% of adult (aged 18 years and over) women and 2.3\% of adult men living with obesity. Uganda's obesity prevalence is lower than the regional average of 20.7\% for women and 9.2\% for men. At the same time, diabetes is estimated to affect 5.6\% of adult women and 5.6\% of adult men \cite{globalnutritionreport}. The motivation for feeding should not only be to stop hunger, but also to increase the general awareness of the nutritional value of the food ingested.

\subsection{Problem statement}
Improper diet and nutrition are caused by inconsistent intake of healthy foods. Other causes include improper meal timings, under or overeating, not having enough healthy foods, and nutritional ignorance. Experts have revealed that only 10\% of children below the age of five years are eating recommended healthy foods and this includes frequenting eating nutritious meals and eating on time. This has left the majority 90\% eating non-nutritious foods which have resulted in increasing numbers of childhood malnutrition and obesity\cite{newvision}. “People are eating a lot of unhealthy foods that don’t contribute to manufacturing of blood in the body which has made them become sicker.” \cite{newvision}. We therefore intend to contain this problem by developing a diet and nutrition management system which will guide the targeted populace categories majorly students to maintain a healthy selection of well-balanced meals daily.

\subsection{Objectives}
\subsubsection{General Objectives}

To develop a diet and nutrition management system that will provide students with timely suggestions on the food they should consume to remain healthy. 

\subsubsection{Specific Objectives}

To identify the requirements for a Diet and Nutrition Management System in order to understand what the the-would-be users of the system would want.

To design a model and implement the system that will be used to assist in maintaining healthy diet meal patterns.

To test and validate the system.  

\subsection{Justification.}
Improper diet and nutrition is a constant challenge to many students which adversely exposes them to risks like obesity, digestive problems, and chronic illnesses. From the listed effects, our system intends to; 
\begin{itemize}
  \item Increase the awareness on the dire consequences of improper diet and nutrition
  \item provide a practical means to improve feeding patterns daily. 
  \item customize a system to the needs of Students Population. 

\end{itemize}

\subsection{Scope}
Our research will be focused on locally grown foods in Uganda that consist a balanced diet and specific chronic illnesses like Diabetes, obesity, pressure and cancer. We will major on Makerere University. We intend to sample several students from colleges such as CEDAT, COCIS, and CHUSS in order to know the foods they regularly eat.

\subsection{Significance}
The system we aim to build will help the elderly remind them when to best consume their meals. Athletes, through consuming the right foods will receive the required food values thus boosting their sports performance. Pregnant mothers will be able to sustain a healthy pregnancy through appropriate meals for both the unborn baby and the expectant mother. Students will be able to register an improvement in their overall mental health, especially regarding memory retention and the intelligence quotient.  People with chronic illnesses will realize the nutritional value of eating suitable foods and as a result, reduced disability-adjusted Life years. As far as academia is concerned, it will assist to refine the focus of the research on the diet and nutrition context and predict any challenges that may arise out of it.


%%-------------------------------------------------------------------------------
%% 2.0 Literature review
%%-------------------------------------------------------------------------------
\section{Literature Review}

According to Wikipedia (2019), dietary management simply means providing nutritional options for individuals and groups with diet concerns through the supervision of food services. Information systems have made it easier for accurate and efficient nutritional analysis of feeding habits and patterns.

This section focuses on the exploration of different solutions developed by researchers to alleviate the problem of poor eating habits among university students. We reviewed similar systems regarding the problems faced by students when they skip meals or even fail to eat at the right time. The solution we aim to provide for this problem is to develop a diet and nutrition management system that will provide students with timely reminders, food suggestions, and nutrition literature.

We realized that so much research has been done by researchers from the field of nutrition science where a person is trained to provide information regarding the types and quantities of food the people eat. This field draws information from other areas such as biology, chemistry, and social sciences (Sriram, 2020).

The review is comprised of works that have been done to find the solution to the problem and we were able to sample various systems. We singled out systems that were closely related to our research problem and we noted the strengths and weaknesses of the solutions. By studying these systems, we were able to find the deficiency in the solutions regarding our problem and stated ways in which the solutions can be tailored to satisfy our target group. The features of the reviewed systems are compared with those of the proposed system in Table 1.

\subsection{Existing Systems}

\subsubsection{MantraCare}

MantraCare is a Ugandan-based web application that provides consultation services with dieticians on belly fat reduction and weight loss dieting plans, exercises, and foods.

\textbf{Strength of the System:}

\begin{itemize}
\item Video calls or chat sessions with the best dieticians.
\item Affordable diet consultation.
\end{itemize}

\textbf{Weaknesses of the System:}

\begin{itemize}
\item It does not have timely reminders.
\item It does not offer food suggestions.
\end{itemize}
\subsubsection{LifeSum}
LifeSum is a fitness app that lets you decide what a healthy lifestyle means to you. It has plenty of recommendations and suggestions, but they are always pointing toward the goal you want.

It has several features which include:
\begin{itemize}
\item Simple tracking of meals (including barcode scanning), exercises, habits, weight, and body measurements.
\item A wide range of diets to choose between, including Keto and High Protein.
\item Favorites - save your favorite food, exercises, meals, and recipes.
\item Meal plans - 1 to 3 weeks of preplanned, easy-to-cook meals.
\item Support for macros and net carbs.
\item Detailed nutritional information.
\item Food, meal, and day ratings.
\item Weekly life score - what's gone well and how you can improve.
\item Hundreds of healthy and tasty recipes.
\item Integrates with Apple Health, Google Fit, Samsung Health, Apple Watch, RunKeeper, Fitbit, Withings, Samsung wearables, Wear OS and Google Assistant.
\end{itemize}

\textbf{Strength of the System:}
\begin{itemize}
\item The ability to integrate with different technologies for example Google Fit, and Apple Watch among others.
\end{itemize}

\textbf{Weaknesses of the System:}
\begin{itemize}
\item Limited food suggestions.
\item Not fully accessible to all people since it's for paying.
\end{itemize}

\subsubsection{My Plate Calorie Counter}
This app provides you with the fastest way to lose weight and improve your overall health.
(\textit{Livestrong.com}) They use the world’s largest food database to provide you with calorie
counts, nutritional information, and serving sizes for lots of foods. You can also access an 8-
week meal plan and recipe suggestions.

\textbf{Strengths of the system:}
\begin{itemize}
\item They provide you with a lot of statistics about the way you eat
\item You can also customize the way you eat depending on your health goals and this includes the weight you wish to gain.
\end{itemize}

\textbf{Weaknesses of the system:}
\begin{itemize}
\item Most of the suggested foods such as Southwestern Pancakes, chocolate Almond protein cocoa, and many others are alien to our target group.
\item If you want to get most of the app such as your highest caloric foods, daily goals, and personal progress you are required to subscribe to the app which 1month costs \$9.99 (approximately 38,000/=)
\item This option would not work for our target group because the suggested foods are hard to find in Uganda and if they are available, they would be so expensive for the students.
\item Most of the students wouldn’t receive the health benefits of the app or any other apps on the market because on average very few Ugandans and especially students don’t appreciate the value of paying for a nutrition app and if they wanted, very few would afford to pay for the app hence making this option is available for very few students.
\end{itemize}

\subsubsection{The Diet Planner Application}

This is one dietary management system that is equipped with features such as 10,000 ready meals, 270 allergens, patients' medical report capability, meal plans, personalized patient menus, nutritional interviews, kitchen and home measures, and a search engine for dishes and products, etc. To save patients' time, specialists create balanced meals that can be enjoyed without restrictions. It will allow for the preparation of nutrition plans for patients faster than ever. It enables the instant generation of fully personalized menus by a wide range of calorie goals, preferences, and dietary restrictions.

\textbf{Strength of the System:}
\begin{itemize}
\item It can generate a health report for the user.
\item It has a search engine for dishes and products.
\end{itemize}

\textbf{Weaknesses of the System:}
\begin{itemize}
\item It does not have timely reminders.
\item It does not have a BMI (Body Mass Index) calculator.
\end{itemize}

\subsubsection{Nutrition and Diet Management Solutions by Nutritics}
This system allows multiple profile details designated as clients to be entered. Each profile is then analyzed and then, a tailored meal plan is suggested according to the profile that has been entered.

\textbf{Strengths of the system.}
\begin{itemize}
\item The system provides support to the users.
\item It has webinars to guide users.
\item It provides multiple meal alternatives.
\item It has multiple categories of users they focus on.
\item It allows the addition of new food and its nutritional values.
\end{itemize}

\textbf{Weaknesses of the System.}
\begin{itemize}
\item The application is customized for desktops only.
\item The application is customized for organizations or hospitals to manage multiple clients/patients.
\item The system is based on expensive meals that may not be affordable.
\end{itemize}

\subsubsection{Noom}
It is a weight loss app that uses a psychology-based approach to change your eating habits for
the better.

\textbf{Strengths of the system.}
\begin{itemize}
\item It has adequate diet nutrition.
\item It is available both on mobile and desktops.
\end{itemize}

\textbf{Weaknesses of the system.}
\begin{itemize}
\item It has so many questions that take almost an hour to be answered.
\item It is not accessible in terms of affordability to people like students.
\end{itemize}

\subsection{Proposed Project}
As a group, we shall develop a web application that will require university students to register possibly with their name and email addresses that will be used to send notifications to eat food cost-effectively. After this, they will be able to receive timely reminders on the minimum amount and type of food they should consume to successfully fulfill their daily tasks. This will include a mix of foods such as Matooke, Rice, and Sweet potatoes and drinks such as water and a cup of tea. It will also include general knowledge on how they can maintain a healthy and functioning body such as drinking lots of water, exercising regularly, and avoiding harmful consumption of alcohol, and drugs among others.
This will provide a more customized and cost-effective feeding way for university students to allow them to focus on their studies as they also take care of their health.

\begin{table}[H]
\centering
\caption{Comparison Of the Systems}
\label{tab:comparison-of -systems}
\resizebox{\columnwidth}{!}{%
\begin{tabular}{@{}|l|l|l|l|l|l|l|l|@{}}
\toprule
\multicolumn{1}{|c|}{\textbf{System Features}} & \textbf{MantraCare} & \textbf{\begin{tabular}[c]{@{}l@{}}Nutrition and Diet \\ Management Solutions\\  By Nutritics\end{tabular}} & \textbf{Noom} & \textbf{\begin{tabular}[c]{@{}l@{}}Diet \\ Planner\end{tabular}} & \textbf{\begin{tabular}[c]{@{}l@{}}My Plate \\ Calories\end{tabular}} & \textbf{Life Sum} & \textbf{\begin{tabular}[c]{@{}l@{}}Proposed \\ System\end{tabular}} \\ \midrule
BMI Calculator &  & \textbf{YES} & \textbf{YES} &  & \textbf{YES} & \textbf{YES} & \textbf{YES} \\ \midrule
Timely Reminder & \textbf{YES} & \textbf{YES} &  &  & \textbf{YES} &  & \textbf{YES} \\ \midrule
Food Suggestions & \textbf{YES} & \textbf{YES} & \textbf{YES} & \textbf{YES} &  &  & \textbf{YES} \\ \midrule
Nutritional Literature & \textbf{YES} &  & \textbf{YES} & \textbf{YES} & \textbf{YES} & \textbf{YES} & \textbf{YES} \\ \midrule
Accessibility(Free) &  &  &  &  &  &  & \textbf{YES} \\ \bottomrule
\end{tabular}%
}
\end{table}

\bibliographystyle{apa7}% Set the bibliography style
\bibliography{citation} % Include the bibliography file
\end{document}
