\documentclass{article}
\usepackage{hyperref}
\begin{document}

\title{TOPIC:  DIET AND NUTRITION MANAGEMENT SYSTEM }
\author{}
\date{\today}
\maketitle 

\section{Introduction}
\label{Introducion}
Information systems are concerned with data capture, storage, analysis, and retrieval. These are vital to assist decision-making in a short time frame, potentially allowing decisions to be made and practices to be actioned in real-time.  In the context of diet and nutrition management, nutritional health systems allow for accurate and efficient analysis of food ingredients as well as patient nutrient intake which assists in ensuring that selected food combinations offer the desired nutrients for a particular meal or diet order. (DFM nutrition management system, n.d.)
\subsection{Background}
Dietary deficiency is more accountable for the overall death toll on the global scale than other factors such as tobacco, high blood pressure, or any other health risk, as indicated in the new scientific study. “Poor diet is an equal opportunity killer,” (IHME, 2019). Low levels of healthy food consumption, such as whole grains, in contrast to too many unhealthy foods, including sweetened beverages, account for one in every five deaths globally, “We are what we eat and risks affect people across a range of demographics, including age, gender, and economic status,” (IHME, 2019). 

Poor diets were behind the 10.9 million deaths, or 22\% of all deaths among adults in 2017, with cardiovascular disease (CVD) as the foremost cause, trailed by cancers and diabetes. They also resulted in 255 million disability-adjusted life years (DALYs), equalling the sum of years lost and years lived with disability. Poor diet statistically represents 16\% of all DALYs among adults globally. The findings of the study dictate that while the impact of individual dietary factors varies across countries, three dietary factors – low intake of whole grains, as well as fruits, and high consumption of sodium – accounted for more than 50\% of diet-related deaths and 66\% of DALYs. The other 50\% of death and 34\% of DALYs were tied to high consumption of red meat, processed meats, sugar-sweetened beverages, and trans fatty acids among other foods. There is an urgent and compelling need for changes in the various sectors of the food production cycle, such as growing, processing, packaging, and marketing (IHME, 2019).

Uganda has shown limited progress toward achieving the diet-related non-communicable disease (NCD) targets. Furthermore, it has shown no progress towards achieving the target for obesity, with an estimated 10.4\% of adult (aged 18 years and over) women and 2.3\% of adult men living with obesity. Uganda's obesity prevalence is lower than the regional average of 20.7\% for women and 9.2\% for men. At the same time, diabetes is estimated to affect 5.6\% of adult women and 5.6\% of adult men (Global Nutrition Report, n.d.). The motivation for feeding should not only be to stop hunger, but also to increase the general awareness of the nutritional value of the food ingested.

\subsection{Problem statement}
Improper diet and nutrition are caused by inconsistent intake of healthy foods. Other causes include improper meal timings, under or overeating, not having enough healthy foods, and nutritional ignorance. Experts have revealed that only 10\% of children below the age of five years are eating recommended healthy foods and this includes frequenting eating nutritious meals and eating on time. This has left the majority 90\% eating non-nutritious foods which have resulted in increasing numbers of childhood malnutrition and obesity (Tumwine, 2022). “People are eating a lot of unhealthy foods that don’t contribute to manufacturing of blood in the body which has made them become sicker.” (Tumwine, 2022).We therefore intend to contain this problem by developing a diet and nutrition management system which will guide the targeted populace categories majorly students to maintain a healthy selection of well-balanced meals daily.

\subsection{Objectives}
\subsubsection{General Objectives}

To develop a diet and nutrition management system that will provide students with timely suggestions on the food they should consume to remain healthy. 

\subsubsection{Specific Objectives}

To identify the requirements for a Diet and Nutrition Management System in order to understand what the the-would-be users of the system would want.

To design a model and implement the system that will be used to assist in maintaining healthy diet meal patterns.

To test and validate the system.  

\subsection{Justification.}
Improper diet and nutrition is a constant challenge to many students which adversely exposes them to risks like obesity, digestive problems, and chronic illnesses. From the listed effects, our system intends to; 
\begin{itemize}
  \item Increase the awareness on the dire consequences of improper diet and nutrition
  \item provide a practical means to improve feeding patterns daily. 
  \item customize a system to the needs of Students Population. 

\end{itemize}

\subsection{Scope}
Our research will be focused on locally grown foods in Uganda that consist a balanced diet and specific chronic illnesses like Diabetes, obesity, pressure and cancer. We will major on Makerere University. We intend to sample several students from colleges such as CEDAT, COCIS, and CHUSS in order to know the foods they regularly eat.

\subsection{Significance}
The system we aim to build will help the elderly remind them when to best consume their meals. Athletes, through consuming the right foods will receive the required food values thus boosting their sports performance. Pregnant mothers will be able to sustain a healthy pregnancy through appropriate meals for both the unborn baby and the expectant mother. Students will be able to register an improvement in their overall mental health, especially regarding memory retention and the intelligence quotient.  People with chronic illnesses will realize the nutritional value of eating suitable foods and as a result, reduced disability-adjusted Life years. As far as academia is concerned, it will assist to refine the focus of the research on the diet and nutrition context and predict any challenges that may arise out of it.
\end{document}
